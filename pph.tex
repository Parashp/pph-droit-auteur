\documentclass[a4paper]{report}
% Fonte & encodage
\renewcommand{\familydefault}{\sfdefault}
\usepackage[utf8]{inputenc}
%\usepackage[T1]{fontenc}
\usepackage[french]{babel}
\usepackage{color}
% Réduire les marges
\usepackage{fullpage}
% Pour pouvoir insérer des images
\usepackage{graphicx}
\graphicspath{images/}
% Pour pouvoir insérer des hyperliens
\usepackage{hyperref}

% Ne veut pas afficher le mot 'Chapitre' avant chaque chapitre
\usepackage{titlesec}
\titleformat{\chapter}[hang]{\bf\huge}{\thechapter}{2pc}{}

% ------------------------------------ %
% -- METADONNÉES DU DOCUMENT --------- %
% Utilisées pour générer la page de garde
\title{
	Quelles solutions pour le droit d'auteur à l'ère d'Internet ?\\
	Projet Personnel en Humanités, INSA Lyon
}
\author{Merlin Nimier-David}
\date{Mai 2014}

\begin{document}

	% Génération de la page de garde
	\maketitle

	% Génération de la table des matières
	\tableofcontents




	% ///////////////////////////////////////////////////////// %
	% /// INTRODUCTION //////////////////////////////////////// %
	\chapter{Introduction}

	Hello introduction.
	% ///////////////////////////////////////////////////////// %




	% ///////////////////////////////////////////////////////// %
	% /// ÉVOLUTION /////////////////////////////////////////// %
	\chapter{Évolution générale du marché de la culture}
	Depuis la démocratisation d'Internet. Vue de globale en se basant sur l'exemple de l'industrie musicale.


	% ------------------------------------ %
	% -- ORGANISATION -------------------- %
	\section{Organisation du marché de la culture}

	% ------------------------------------ %

	% ------------------------------------ %
	% -- LA RÉVOLUTION NAPSTER ----------- %
	\section{La révolution Napster}

	% ------------------------------------ %

	% ------------------------------------ %
	% -- UN SYSTÈME SUR LE DÉCLIN -------- %
	\section{Un système sur le déclin}
	Crise du disque, baisse des revenus.
	% ------------------------------------ %

	% ///////////////////////////////////////////////////////// %




	% ///////////////////////////////////////////////////////// %
	% /// SOLUTIONS ACTUELLES ///////////////////////////////// %
	\chapter{Solutions actuelles}
	Examiner les différentes solutions mises en place et estimer leur efficacité.

	% ------------------------------------ %
	% -- NOUVEAUX MODÈLES ÉCONOMIQUE ----- %
	\section{Nouveaux modèles économique}
	\begin{enumerate}
		\item Distribution numérique (facilité, prix, DRM, pas de hausse de la rémunération malgré les coûts de distribution réduits)
		\item Freemium (Deezer, etc : très faible rémunération)
		\item Offres illimitées
	\end{enumerate}
	% ------------------------------------ %

	% ------------------------------------ %
	% -- TENTATIVES DE RÉGULATION ----- %
	\section{Tentatives de régulation}
	\begin{enumerate}
		\item Échec des DRM
		\item Hadopi
		\item SOPA/PIPA
	\end{enumerate}
	% ------------------------------------ %

	% ///////////////////////////////////////////////////////// %




	% ///////////////////////////////////////////////////////// %
	% /// PISTES ALTERNATIVES DE RÉSOLUTION /////////////////// %
	\chapter{Pistes alternatives de résolution}

	\begin{enumerate}
		\item Contournement des majors : explosion du marché indépendant
		\begin{enumerate}
		 	\item financement participatif
		 	\item distribution directe
		 \end{enumerate}
		 \item Diversification de l'offre légale
		 \item Quel rôle pour l'État ? (licence globale ?)
	\end{enumerate}

	% ///////////////////////////////////////////////////////// %




	% ///////////////////////////////////////////////////////// %
	% /// CONCLUSION ////////////////////////////////////////// %
	\chapter{Conclusion}

	Bye.
	% ///////////////////////////////////////////////////////// %




% Fin du document
\end{document}