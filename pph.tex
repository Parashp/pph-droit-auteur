\documentclass[a4paper]{report}
% Fonte & encodage
\renewcommand{\familydefault}{\sfdefault}
\usepackage[utf8]{inputenc}
%\usepackage[T1]{fontenc}
\usepackage[french]{babel}
\usepackage{color}
% Réduire les marges
%\usepackage{fullpage}
% Pour pouvoir insérer des images
\usepackage{graphicx}
\graphicspath{images/}
% Pour pouvoir insérer des hyperliens
\usepackage{hyperref}

% Ne veut pas afficher le mot 'Chapitre' avant chaque chapitre
\usepackage{titlesec}
\titleformat{\chapter}[hang]{\bf\huge}{\thechapter}{2pc}{}

% ------------------------------------ %
% -- METADONNÉES DU DOCUMENT --------- %
% Utilisées pour générer la page de garde
\title{
	Quelles solutions pour le droit d'auteur à l'ère d'Internet ?\\
	Projet Personnel en Humanités, INSA Lyon
}
\author{Merlin Nimier-David}
\date{Mai 2014}

\begin{document}

	% Génération de la page de garde
	\maketitle

	% Génération de la table des matières
	\tableofcontents

	\chapter{TODO}
	\begin{enumerate}
		\item Bibliographie (fichier \texttt{.bib})
		\item Source de chaque figure
		\item Remplacer les titres ``in picture'' par des \texttt{caption}
		\item Remplacer les footnotes (``Source : '') par des références de la bibliographie.
		\item Insérer des images d'agrément
		\item Bien centrer les graphiques
		\item Actualiser la partie \ref{declin-systeme}{Système sur le déclin} : l'industrie musicale atteint l'équilibre (citer le podcast — Pascal Nègre)
	\end{enumerate}


	% ///////////////////////////////////////////////////////// %
	% /// INTRODUCTION //////////////////////////////////////// %
	\chapter{Introduction}
	Il n'a jamais été aussi facile d'accéder à une œuvre qu'aujourd'hui. Sans connaissance technique particulière et simplement équipé d'un accès à internet, n'importe que morceau de musique est accessible. Cette affirmation est également vraie pour les films, séries, livres, jeux vidéo et logiciels. L'accès universel à la culture a changé notre manière de la percevoir — tout du moins en ce qui concerne les \emph{digital natives}. Cette culture est désormais réellement \emph{populaire}, c'est-à-dire accessible à tous, sans barrière financière. Mais comment assurer la pérennité économique d'une industrie entière alors que ses produits pourraient avoir perdu toute valeur monétaire aux yeux de ses consommateurs ?

	Nous adoptons ici un point de vue économique centré sur la question du droit d'auteur, laissant volontairement de côté les aspects juridiques ou éthiques associés. De même, on se concentrera sur les œuvre enregistrées et distribuables et n'inclurons donc pas le spectacle vivant, les peintures, sculptures, etc. En première partie, nous présentons une vue simplifiée du marché de la culture et donnons un historique rapide de son évolution récente. Nous abordons ensuite les solutions ayant été expérimentées. Enfin, nous tenterons de mettre en avant plusieurs solutions alternatives.

	% ///////////////////////////////////////////////////////// %




	% ///////////////////////////////////////////////////////// %
	% /// ÉVOLUTION /////////////////////////////////////////// %
	\chapter{Évolution générale du marché de la culture}
	Depuis la démocratisation d'Internet. Vue de globale en se basant sur l'exemple de l'industrie musicale.


	% ------------------------------------ %
	% -- ORGANISATION -------------------- %
	\section{Organisation du marché de la culture}

	% ------------------------------------ %

	% ------------------------------------ %
	% -- LA RÉVOLUTION NAPSTER ----------- %
	\section{La révolution Napster}

	% ------------------------------------ %

	% ------------------------------------ %
	% -- UN SYSTÈME SUR LE DÉCLIN -------- %
	\section{Un système sur le déclin}
	Crise du disque, baisse des revenus.
	% ------------------------------------ %

	% ///////////////////////////////////////////////////////// %




	% ///////////////////////////////////////////////////////// %
	% /// SOLUTIONS ACTUELLES ///////////////////////////////// %
	\chapter{Solutions actuelles}
	Examiner les différentes solutions mises en place et estimer leur efficacité.

	% ------------------------------------ %
	% -- NOUVEAUX MODÈLES ÉCONOMIQUE ----- %
	\section{Nouveaux modèles économique}
	\begin{enumerate}
		\item Distribution numérique (facilité, prix, DRM, pas de hausse de la rémunération malgré les coûts de distribution réduits)
		\item Freemium (Deezer, etc : très faible rémunération)
		\item Offres illimitées
	\end{enumerate}
	% ------------------------------------ %

	% ------------------------------------ %
	% -- TENTATIVES DE RÉGULATION ----- %
	\section{Tentatives de régulation}
	\begin{enumerate}
		\item Échec des DRM
		\item Hadopi
		\item SOPA/PIPA
	\end{enumerate}
	% ------------------------------------ %

	% ///////////////////////////////////////////////////////// %




	% ///////////////////////////////////////////////////////// %
	% /// PISTES ALTERNATIVES DE RÉSOLUTION /////////////////// %
	\chapter{Pistes alternatives de résolution}

	\begin{enumerate}
		\item Contournement des majors : explosion du marché indépendant
		\begin{enumerate}
		 	\item financement participatif
		 	\item distribution directe
		 \end{enumerate}
		 \item Diversification de l'offre légale
		 \item Quel rôle pour l'État ? (licence globale ?)
	\end{enumerate}

	% ///////////////////////////////////////////////////////// %




	% ///////////////////////////////////////////////////////// %
	% /// CONCLUSION ////////////////////////////////////////// %
	\chapter{Conclusion}

	Bye.
	% ///////////////////////////////////////////////////////// %




% Fin du document
\end{document}